\newcommand{\letitle}{}
\newcommand{\leauthor}{}

\documentclass[a4paper,12pt]{article}
\usepackage[utf8]{inputenc}
\usepackage{amssymb}
\usepackage{amsmath}
\usepackage[margin=1.5cm]{geometry}


\everymath{\displaystyle}
\title{\letitle}
\author{\leauthor}
\date{}
\begin{document}
    \section{THEORIE GENERALE}

\subsection{DETERMINANT DANS UNE BASE}

\subsubsection{Notion de forme p-lineaire sur E.}

L'application $\varphi:E^p \rightarrow \mathbb{K}$
 est {\it p-lineaire} si chacune des applications partielles
 est lineaire.

\subsubsection{Forme alternée, antisymétrique.}

On suppose que $\varphi$ est une forme p-linéaire sur E.

\begin{itemize}
\item Définitions

$\varphi$ est \textbf{ alternée } si $\varphi$ s’annule sur tout système de p vecteurs contenant au moins
deux vecteurs égaux. On dit qu’elle est \textbf{ antisymétrique } si, à chaque fois que le système S'
est déduit du système S par permutation de deux vecteurs, on a : $\varphi(S')=-\varphi(S)$.

\item Propriétés

\begin{enumerate}
\item $\varphi$ alternée $\implies \varphi$ antisymétrique. Réciproque vraie si car $\mathbb{K}\neq 2$

\item $\varphi$ antisymétrique $\iff \forall \sigma \in S_p$, $\forall (x_1,\hdots,x_p) 
\in E^p, \varphi(x_{\sigma(1)},\hdots,x_{\sigma(p)})=\epsilon (\sigma).\varphi(x_1,\hdots,x_p)$.

\item $\varphi$ alternée $\iff \forall (x_1,\hdots,x_p)\in E^p, ((x_1,\hdots,x_p)$ 
lié $\implies \varphi (x_1,\hdots,x_p) = 0)$.
\end{enumerate}
\end{itemize}

\subsubsection{\textbf{Théorème :}}
Soit E de dimension n, et B une base de E. Il existe une unique forme n-
linéaire alternée $\varphi$ sur E valant 1 sur la base B. Par définition, $\varphi$ = det B . On a de plus la
formule :
\begin{equation}\forall x \in E, det_B(x_1,\hdots,x_n)=\sum_{\sigma \in \mathcal{S}} 
\epsilon (\sigma).\varphi_{\sigma(1)}(x_1)\hdots \varphi_{\sigma(n)}(x_n).\end{equation}
où $(\varphi_1,\hdots,\varphi_n)$ est la pase duale de $(e_1,\hdots,e_n)$.

\end{document}
    